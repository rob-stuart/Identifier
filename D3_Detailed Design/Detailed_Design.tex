\documentclass[titlepage,12pt]{article}

% Imported Packages
%------------------------------------------------------------------------------
\usepackage{amssymb}
\usepackage{amstext}
\usepackage{amsthm}
\usepackage{amsmath}
\usepackage{enumerate}
\usepackage{fancyhdr}
\usepackage[margin=1in]{geometry}
\usepackage{graphicx}
\usepackage{extarrows}
\usepackage{setspace}

\usepackage{array}
\usepackage{color}
\usepackage[hidelinks]{hyperref}
\usepackage{float}
%------------------------------------------------------------------------------

% Header and Footer
%------------------------------------------------------------------------------
\pagestyle{plain}  
\renewcommand\headrulewidth{0.4pt}                                      
\renewcommand\footrulewidth{0.4pt}                                    
%------------------------------------------------------------------------------

% Title Details
%------------------------------------------------------------------------------
\title{ParkFinder\\Detailed Design Document\\SE 3A04}
\author{Abdul Ahad \\ akhteraa \and Salma Belal \\ belalsm \and Josh Chatten \\ chattejj \and
	Nathanael Jordan \\ jordanen \and Robert Stuart \\ stuarr2}
\date{\today}                               
%------------------------------------------------------------------------------

% Document
%------------------------------------------------------------------------------
\begin{document}

\maketitle	
\thispagestyle{empty}
\clearpage
\setcounter{tocdepth}{2}% Only show down to subsections?
\tableofcontents
\clearpage


\section{Introduction}
\label{sec:introduction}
% Begin Section

This section provides an brief overview of the entire document.

\subsection{Purpose}
\label{sub:purpose}
The purpose of this Detailed Design Document is to provide a description for the detailed design of
the ParkFinder app. The description of the design will allow anyone who will be involved in the
development of the system to proceed with an understanding of what is to be built and how it is
expected to be built. This document provides a description of the system's classes and static
structure, as well as diagrams that describe the dynamic behaviour of a system in response to
external stimuli, and diagrams that describe the interactions among classes in terms of an exchange
of messages over time.\\

The intended readers of this document include all of the project's stakeholders. This includes the
end-user, the software engineers, and the park authorities.

\subsection{System Description}
\label{sub:system_description}
The software system being described in this document is called the ParkFinder app. This system will
have datasets of information about parks from all over the world and will allow the client to use
search methods in order to find parks based on the clients' desired attributes. The app is meant to
be used anywhere in the world, provided an Android or iOS device with the app installed. This
provides clients with an easier, faster, and more efficient way to look up parks and acquire
information such as the location, facilities, activities, and rentals that the parks provide.

\subsection{Overview}
\label{sub:overview}
% Begin SubSection
The remainder of this document will contain diagrams and information that will describe the details
for the software system being built. This will include State Charts for controller classes in
Section~\ref{sec:state_charts_for_controller_classes}, Sequence Diagrams in Section~\ref
{sec:sequence_diagrams}, and a Detailed Class Diagram in Section~\ref{sec:detailed_class_diagram}.

% End SubSection

% End Section

\section{State Charts for Controller Classes}
\label{sec:state_charts_for_controller_classes}
% Begin Section
This section should provide a state chart for each controller class for your application.

% \begin{figure}[H]
%     \centerline{\includegraphics[width=0.90\textwidth]{state_diagrams/a_file_name}}
%     \caption{A Caption} % Note: don't need to add extension to the fileName ^^^^
%     \label{a_label}
% \end{figure}

% End Section

\section{Sequence Diagrams}
\label{sec:sequence_diagrams}
% Begin Section
This section should provide a sequence diagram for each use case of your application.

% \begin{figure}[H]
%     \centerline{\includegraphics[width=0.90\textwidth]{sequence_diagrams/a_file_name}}
%     \caption{A Caption} % Note: don't need to add extension to the fileName ^^^^
%     \label{a_label}
% \end{figure}

% End Section

\section{Detailed Class Diagram}
\label{sec:detailed_class_diagram}
% Begin Section
This section should provide a detailed class diagram for your application.

% \begin{figure}[H]
%     \centerline{\includegraphics[width=0.90\textwidth]{class_diagrams/a_file_name}}
%     \caption{A Caption} % Note: don't need to add extension to the fileName ^^^^
%     \label{a_label}
% \end{figure}

% End Section

\appendix
\section{Division of Labour}
\label{sec:division_of_labour}
% Begin Section

\begin{table}[H]
\vspace{-0.06in}
\begin{center}
\setlength{\extrarowheight}{4.0pt}
\begin{tabular}{m{0.3\textwidth} m{0.2\textwidth} m{0.3\textwidth}} 
\hline
\textbf{Contributions} & \textbf{Name} & \textbf{Signature}\\
\hline
Section~\ref{sec:sequence_diagrams} & Abdul Ahad & \\
\hline
Section~\ref{sec:introduction} & Salma Belal & \\
\hline
Section~\ref{sec:state_charts_for_controller_classes} & Josh Chatten & \\
\hline
Section~\ref{sec:state_charts_for_controller_classes} & Nathanael Jordan  & \\
\hline
Section~\ref{sec:detailed_class_diagram} & Robert Stuart & \\
\hline
\end{tabular}
\end{center}
\label{divOfLabour}
\end{table}

% End Section

\newpage
\section*{IMPORTANT NOTES}
\begin{itemize}
    \item You do \underline{NOT} need to provide a text explanation of each diagram; the diagram
    should speak for itself 
    \item Please document any non-standard notations that you may have used
	\begin{itemize}
        \item \emph{Rule of Thumb}: if you feel there is any doubt surrounding the meaning of your
        notations, document them
	\end{itemize}
    \item Some diagrams may be difficult to fit into one page
	\begin{itemize}
        \item It is OK if the text is small but please ensure that it is readable when printed
        \item If you need to break a diagram onto multiple pages, please adopt a system of doing so
        and throughly explain how it can be reconnected from one page to the next; if you are unsure
        about this, please ask me
	\end{itemize}
    \item Please submit the latest version of Deliverable 1 and Deliverable 2 with Deliverable 3
	\begin{itemize}
        \item They do not have to be a freshly printed versions; the latest marked versions are OK
	\end{itemize}
    \item If you do \underline{NOT} have a Division of Labour sheet, your deliverable will
    \underline{NOT} be marked
\end{itemize}


\end{document}
%------------------------------------------------------------------------------