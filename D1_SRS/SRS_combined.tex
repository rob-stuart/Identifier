\documentclass[titlepage]{article}

% Imported Packages
%------------------------------------------------------------------------------
\usepackage{amssymb}
\usepackage{amstext}
\usepackage{amsthm}
\usepackage{amsmath}
\usepackage{enumerate}
\usepackage{fancyhdr}
\usepackage[margin=1in]{geometry}
\usepackage{graphicx}
\usepackage{extarrows}
\usepackage{setspace}

\usepackage{array}
\usepackage{color}
\usepackage[hidelinks]{hyperref}

%------------------------------------------------------------------------------

% Header and Footer
%------------------------------------------------------------------------------
\pagestyle{plain}  
\renewcommand\headrulewidth{0.4pt}                                      
\renewcommand\footrulewidth{0.4pt}                                    
%------------------------------------------------------------------------------

% Title Details
%------------------------------------------------------------------------------
\title{ParkFinder\\Software Requirements Document\\SE 3A04}
\author{Abdul Ahad \\ \textcolor{red}{??????} \and Salma Belal \\ belalsm \and Josh Chatten \\
chattejj \and Nathanael Jordan \\ jordanen \and Robert Stuart \\ stuarr2}
\date{\today}

%------------------------------------------------------------------------------

% Document
%------------------------------------------------------------------------------
\begin{document}

\maketitle	
\thispagestyle{empty}
\clearpage
\setcounter{tocdepth}{2}% Only show down to subsections?
\tableofcontents
\clearpage

\section{Introduction}%TODO
\label{sec:introduction}
% Begin Section

This section of the SRS should provide an overview of the entire SRS.

\subsection{Purpose}%TODO
\label{sub:purpose}
% Begin SubSection
\begin{enumerate}[a)]
	\item Delineate the purpose of the SRS
	\item Specify the intended audience for the SRS
\end{enumerate}
% End SubSection

\subsection{Scope}%TODO
\label{sub:scope}
% Begin SubSection
\begin{enumerate}[a)]
    \item Identify the software product(s) to be produced by name (e.g., Host DBMS, Report
    Generator, etc.)
	\item Explain what the software product(s) will, and, if necessary, will not do
	\item Describe the application of the software being specified, including relevant benefits,
	objectives, and goals
	\item Be consistent with similar statements in higher-level specifications (e.g., the system
	requirements specification), if they exist 
\end{enumerate}
% End SubSection

\subsection{Definitions, Acronyms, and Abbreviations}%TODO
\label{sub:definitions_acronyms_and_abbreviations}
% Begin SubSection
\begin{description}
    \item[\textbf{Experts}] A search criteria used for the identification of a particular park, or
    group of parks. ex. An expert could be for the rentals available at a park
\end{description}


\color{red}
\begin{enumerate}[a)]
	\item Provide the definitions of all terms, acronyms, and abbreviations required to properly
	interpret the SRS
\end{enumerate}
\color{black}
% End SubSection

\subsection{References}%TODO
\label{sub:references}
% Begin SubSection
\begin{enumerate}[a)]
	\item Provide a complete list of all documents referenced elsewhere in the SRS
    \item Identify each document by title, report number (if applicable), date, and publishing
    organization
	\item Specify the sources from which the references can be obtained
\end{enumerate}
% End SubSection

\subsection{Overview}%TODO
\label{sub:overview}
% Begin SubSection
\begin{enumerate}[a)]
	\item Describe what the rest of the SRS contains
	\item Explain how the SRS is organized
\end{enumerate}
% End SubSection

% End Section

\section{Overall Description}%TODO
\label{sec:overall_description}
% Begin Section

\subsection{Product Perspective}%TODO
\label{sub:product_perspective}
% Begin SubSection
Several web-services, such as Google or \url{www.ontarioparks.ca}, are available to assist with
locating parks. However these services are not available to an off-line user. The ParkFinder app
will allow its users to find parks that match their search criteria. More importantly, the app
will be able to perform the majority of its search functionalities without an Internet connection,
thus allowing the user the freedom to use the app whenever and wherever they want. As the ParkFinder
app only requires external resources for a subset of its functionality and no external applications
depend upon the ParkFinder app, it cannot be considered to be a component of a larger system.
% End SubSection

\subsection{Product Functions}%TODO
\label{sub:product_functions}
% Begin SubSection
\begin{enumerate}[a)]
	\item The product shall maintain a database of parks and associated attributes
    \item The app will provide a minimum of four methods for querying the park database, these
    methods	will be referred to as ``experts''
    \item The experts must be separate from one another, and be easily swappable
    \item One expert must be able to locate parks based on location, to do so
    an on-line mapping service will need to be utilized (when an Internet connection is available)
    \item Results from a users query shall be displayed via a centralized ``forum''
    \item Messages sent between the user and the experts shall be encrypted
    \item The user can select a park from the query results and the ParkFinder app shall display
    more detailed information about that park, including a weather report (when an Internet
    connection is available)
\end{enumerate}
% End SubSection

\subsection{User Characteristics}%TODO
\label{sub:user_characteristics}
% Begin SubSection
The intended users of the ParkFinder app are people wishing to discover new parks that they have not
been to before as well as people completely new to visiting parks. It is expected that the primary
users will be adults however, visiting parks can be a family affair and thus children can also
be expected to use the ParkFinder. Thus we can expect the ParkFinder app to be used by children,
adults, and the elderly. With such a large age range, only the most basic education levels can be
expected. Being an app solely available on mobile devices, it can be expected that users will
possess the bare minimum skill required to operate a mobile device. Such skills would include being
able to select buttons, use a keyboard, and navigate menu screens.
% End SubSection

\subsection{Constraints}%TODO
\label{sub:constraints}
% Begin SubSection
The primary constraint limiting the development teams options is time. Due to chaotic scheduling
between this project and others, some minor functionality may not be implemented.
% End SubSection

\subsection{Assumptions and Dependencies}%TODO
\label{sub:assumptions_and_dependencies}
% Begin SubSection
N/A?


\color{red}
\begin{enumerate}[a)]
    \item List each of the factors that affect the requirements stated in the SRS
    \item These factors are not design constraints on the software but are, rather, any changes to
    them that can affect the requirements in the SRS
	\begin{itemize}
        \item \textbf{Example}: An assumption may be that a specific operating system will be
        available on the hardware designated for the software product. If, in fact, the operating
        system is not available, the SRS would then have to change accordingly.
	\end{itemize}
\end{enumerate}
\color{black}
% End SubSection

\subsection{Apportioning of Requirements}%TODO
\label{sub:apportioning_of_requirements}
% Begin SubSection
N/A?


\color{red}
\begin{enumerate}[a)]
	\item Identify requirements that may be delayed until future versions of the system
\end{enumerate}
\color{black}
% End SubSection

% End Section

\section{Functional Requirements}%TODO
\label{sec:functional_requirements}
% Begin Section
This section of the SRS should contain all of the software requirements to a level of detail
sufficient to enable designers to design a system to satisfy those requirements, and testers to test
that the system satisfies those requirements. Throughout this section, every stated requirement
should be externally perceivable by users, operators, or other external systems. These requirements
should include at a minimum a description of every input (stimulus) into the system, every output
(response) from the system, and all functions performed by the system in response to an input or in
support of an output.

You normally have two options for organizing your functional requirements:
\begin{enumerate}
	\item Organize first by \emph{business events}, then by \emph{viewpoints}
	\item Organize first by \emph{viewpoints}, then by \emph{business events}
\end{enumerate}
Choose the one which makes the most sense.

For example, if you wish to organization by business events:
\begin{enumerate}[{BE}1.]
	\item Business Event
	\begin{enumerate}[{VP1}.1]
		\item Viewpoint
			\begin{enumerate}
				\item Requirement
				\item Requirement
				\item \dots
			\end{enumerate}
		\item Viewpoint
			\begin{enumerate}
				\item Requirement
				\item Requirement
				\item \dots
			\end{enumerate}
		\item \dots
	\end{enumerate}
	\item Business Event
	\begin{enumerate}[{VP2}.1]
		\item Viewpoint
			\begin{enumerate}
				\item Requirement
				\item Requirement
				\item \dots
			\end{enumerate}
		\item Viewpoint
			\begin{enumerate}
				\item Requirement
				\item Requirement
				\item \dots
			\end{enumerate}
		\item \dots
	\end{enumerate}
\end{enumerate}

\underline{OR}, if you wish to organization by viewpoints:
\begin{enumerate}[{VP}1.]
	\item Viewpoint 
	\begin{enumerate}[{BE1}.1]
		\item Business Event
		\begin{enumerate}
			\item Requirement
			\item Requirement
			\item \dots
		\end{enumerate}
		\item Business Event
		\begin{enumerate}
			\item Requirement
			\item Requirement
			\item \dots
		\end{enumerate}
		\item \dots
	\end{enumerate}
	\item Viewpoint
	\begin{enumerate}[{BE2}.1]
		\item Business Event
		\begin{enumerate}
			\item Requirement
			\item Requirement
			\item \dots
		\end{enumerate}
		\item Business Event
		\begin{enumerate}
			\item Requirement
			\item Requirement
			\item \dots
		\end{enumerate}
		\item \dots
	\end{enumerate}
\end{enumerate}

% End Section

\section{Non-Functional Requirements}%TODO
\label{sec:non-functional_requirements}
% Begin Section
\subsection{Look and Feel Requirements}%TODO
\label{sub:look_and_feel_requirements}
% Begin SubSection

\subsubsection{Appearance Requirements}%TODO
\label{ssub:appearance_requirements}
% Begin SubSubSection
\begin{enumerate}[{LF}1. ]
	\item 
\end{enumerate}
% End SubSubSection

\subsubsection{Style Requirements}%TODO
\label{ssub:style_requirements}
% Begin SubSubSection
\begin{enumerate}[{LF}1. ]
	\item 
\end{enumerate}
% End SubSubSection

% End SubSection

\subsection{Usability and Humanity Requirements}%TODO
\label{sub:usability_and_humanity_requirements}
% Begin SubSection

\subsubsection{Ease of Use Requirements}%TODO
\label{ssub:ease_of_use_requirements}
% Begin SubSubSection
\begin{enumerate}[{UH}1. ]
	\item 
\end{enumerate}
% End SubSubSection

\subsubsection{Personalization and Internationalization Requirements}%TODO
\label{ssub:personalization_and_internationalization_requirements}
% Begin SubSubSection
\begin{enumerate}[{UH}1. ]
	\item 
\end{enumerate}
% End SubSubSection

\subsubsection{Learning Requirements}%TODO
\label{ssub:learning_requirements}
% Begin SubSubSection
\begin{enumerate}[{UH}1. ]
	\item 
\end{enumerate}
% End SubSubSection

\subsubsection{Understandability and Politeness Requirements}%TODO
\label{ssub:understandability_and_politeness_requirements}
% Begin SubSubSection
\begin{enumerate}[{UH}1. ]
	\item 
\end{enumerate}
% End SubSubSection

\subsubsection{Accessibility Requirements}%TODO
\label{ssub:accessibility_requirements}
% Begin SubSubSection
\begin{enumerate}[{UH}1. ]
	\item 
\end{enumerate}
% End SubSubSection

% End SubSection

\subsection{Performance Requirements}%TODO
\label{sub:performance_requirements}
% Begin SubSection

\subsubsection{Speed and Latency Requirements}%TODO
\label{ssub:speed_and_latency_requirements}
% Begin SubSubSection
\begin{enumerate}[{PR}1. ]
	\item 
\end{enumerate}
% End SubSubSection

\subsubsection{Safety-Critical Requirements}%TODO
\label{ssub:safety_critical_requirements}
% Begin SubSubSection
\begin{enumerate}[{PR}1. ]
	\item 
\end{enumerate}
% End SubSubSection

\subsubsection{Precision or Accuracy Requirements}%TODO
\label{ssub:precision_or_accuracy_requirements}
% Begin SubSubSection
\begin{enumerate}[{PR}1. ]
	\item 
\end{enumerate}
% End SubSubSection

\subsubsection{Reliability and Availability Requirements}%TODO
\label{ssub:reliability_and_availability_requirements}
% Begin SubSubSection
\begin{enumerate}[{PR}1. ]
	\item 
\end{enumerate}
% End SubSubSection

\subsubsection{Robustness or Fault-Tolerance Requirements}%TODO
\label{ssub:robustness_or_fault_tolerance_requirements}
% Begin SubSubSection
\begin{enumerate}[{PR}1. ]
	\item 
\end{enumerate}
% End SubSubSection

\subsubsection{Capacity Requirements}%TODO
\label{ssub:capacity_requirements}
% Begin SubSubSection
\begin{enumerate}[{PR}1. ]
	\item 
\end{enumerate}
% End SubSubSection

\subsubsection{Scalability or Extensibility Requirements}%TODO
\label{ssub:scalability_or_extensibility_requirements}
% Begin SubSubSection
\begin{enumerate}[{PR}1. ]
	\item 
\end{enumerate}
% End SubSubSection

\subsubsection{Longevity Requirements}%TODO
\label{ssub:longevity_requirements}
% Begin SubSubSection
\begin{enumerate}[{PR}1. ]
	\item 
\end{enumerate}
% End SubSubSection

% End SubSection

\subsection{Operational and Environmental Requirements}%TODO
\label{sub:operational_and_environmental_requirements}
% Begin SubSection

\subsubsection{Expected Physical Environment}%TODO
\label{ssub:expected_physical_environment}
% Begin SubSubSection
\begin{enumerate}[{OE}1. ]
	\item 
\end{enumerate}
% End SubSubSection

\subsubsection{Requirements for Interfacing with Adjacent Systems}%TODO
\label{ssub:requirements_for_interfacing_with_adjacent_systems}
% Begin SubSubSection
\begin{enumerate}[{OE}1. ]
	\item 
\end{enumerate}
% End SubSubSection

\subsubsection{Productization Requirements}%TODO
\label{ssub:productization_requirements}
% Begin SubSubSection
\begin{enumerate}[{OE}1. ]
	\item 
\end{enumerate}
% End SubSubSection

\subsubsection{Release Requirements}%TODO
\label{ssub:release_requirements}
% Begin SubSubSection
\begin{enumerate}[{OE}1. ]
	\item 
\end{enumerate}
% End SubSubSection

% End SubSection

\subsection{Maintainability and Support Requirements}%TODO
\label{sub:maintainability_and_support_requirements}
% Begin SubSection

\subsubsection{Maintenance Requirements}%TODO
\label{ssub:maintenance_requirements}
% Begin SubSubSection
\begin{enumerate}[{MS}1. ]
	\item 
\end{enumerate}
% End SubSubSection

\subsubsection{Supportability Requirements}%TODO
\label{ssub:supportability_requirements}
% Begin SubSubSection
\begin{enumerate}[{MS}1. ]
	\item 
\end{enumerate}
% End SubSubSection

\subsubsection{Adaptability Requirements}%TODO
\label{ssub:adaptability_requirements}
% Begin SubSubSection
\begin{enumerate}[{MS}1. ]
	\item 
\end{enumerate}
% End SubSubSection

% End SubSection

\subsection{Security Requirements}%TODO
\label{sub:security_requirements}
% Begin SubSection

\subsubsection{Access Requirements}%TODO
\label{ssub:access_requirements}
% Begin SubSubSection
\begin{enumerate}[{SR}1. ]
	\item 
\end{enumerate}
% End SubSubSection

\subsubsection{Integrity Requirements}%TODO
\label{ssub:integrity_requirements}
% Begin SubSubSection
\begin{enumerate}[{SR}1. ]
	\item 
\end{enumerate}
% End SubSubSection

\subsubsection{Privacy Requirements}%TODO
\label{ssub:privacy_requirements}
% Begin SubSubSection
\begin{enumerate}[{SR}1. ]
	\item 
\end{enumerate}
% End SubSubSection

\subsubsection{Audit Requirements}%TODO
\label{ssub:audit_requirements}
% Begin SubSubSection
\begin{enumerate}[{SR}1. ]
	\item 
\end{enumerate}
% End SubSubSection

\subsubsection{Immunity Requirements}%TODO
\label{ssub:immunity_requirements}
% Begin SubSubSection
\begin{enumerate}[{SR}1. ]
	\item 
\end{enumerate}
% End SubSubSection

% End SubSection

\subsection{Cultural and Political Requirements}%TODO
\label{sub:cultural_and_political_requirements}
% Begin SubSection

\subsubsection{Cultural Requirements}%TODO
\label{ssub:cultural_requirements}
% Begin SubSubSection
\begin{enumerate}[{CP}1. ]
	\item 
\end{enumerate}
% End SubSubSection

\subsubsection{Political Requirements}%TODO
\label{ssub:political_requirements}
% Begin SubSubSection
\begin{enumerate}[{CP}1. ]
	\item 
\end{enumerate}
% End SubSubSection

% End SubSection

\subsection{Legal Requirements}%TODO
\label{sub:legal_requirements}
% Begin SubSection

\subsubsection{Compliance Requirements}%TODO
\label{ssub:compliance_requirements}
% Begin SubSubSection
\begin{enumerate}[{LR}1. ]
	\item 
\end{enumerate}
% End SubSubSection

\subsubsection{Standards Requirements}%TODO
\label{ssub:standards_requirements}
% Begin SubSubSection
\begin{enumerate}[{LR}1. ]
	\item 
\end{enumerate}
% End SubSubSection

% End SubSection

% End Section
\newpage
\appendix
\section{Division of Labour}%TODO
\label{sec:division_of_labour}
% Begin Section
\color{red}
Include a Division of Labour sheet which indicates the contributions of each team member. This sheet
must be signed by all team members.
\color{black}

\begin{table}[htbp]
\caption{Division of Labour}
\vspace{-0.06in}
\begin{center}
\setlength{\extrarowheight}{4.0pt}
\begin{tabular}{m{0.3\textwidth} m{0.2\textwidth} m{0.3\textwidth}} 
\hline
\textbf{Contributions} & \textbf{Name} & \textbf{Signature}\\
\hline
Section~\ref{sec:non-functional_requirements} & Abdul Ahad & \\
\hline
Section~\ref{sec:functional_requirements} & Salma Belal & \\
\hline
Section~\ref{sec:introduction} & Josh Chatten & \\
\hline
Section~\ref{sec:non-functional_requirements} & Nathanael Jordan  & \\
\hline
Section~\ref{sec:overall_description} & Robert Stuart & \\
\hline
\end{tabular}
\end{center}
\label{divOfLabour}
\end{table}
% End Section
\end{document}
%------------------------------------------------------------------------------