\documentclass[titlepage]{article}

% Imported Packages
%------------------------------------------------------------------------------
\usepackage{amssymb}
\usepackage{amstext}
\usepackage{amsthm}
\usepackage{amsmath}
\usepackage{enumerate}
\usepackage{fancyhdr}
\usepackage[margin=1in]{geometry}
\usepackage{graphicx}
\usepackage{extarrows}
\usepackage{setspace}

\usepackage{array}
\usepackage{color}
\usepackage[hidelinks]{hyperref}
%------------------------------------------------------------------------------

% Custom commands
%------------------------------------------------------------------------------
\newcounter{myCounter}

%------------------------------------------------------------------------------

% Header and Footer
%------------------------------------------------------------------------------
\pagestyle{plain}
\renewcommand\headrulewidth{0.4pt}
\renewcommand\footrulewidth{0.4pt}
%------------------------------------------------------------------------------

% Title Details
%------------------------------------------------------------------------------
\title{ParkFinder\\Software Requirements Document\\SE 3A04}
\author{Abdul Ahad \\ akhteraa \and Salma Belal \\ belalsm \and Josh Chatten \\
chattejj \and Nathanael Jordan \\ jordanen \and Robert Stuart \\ stuarr2}
\date{\today}

%------------------------------------------------------------------------------

% Document
%------------------------------------------------------------------------------
\begin{document}

\maketitle	
\thispagestyle{empty}
\clearpage
\setcounter{tocdepth}{2}% Only show down to subsections?
\tableofcontents
\clearpage

\section{Introduction}
\label{sec:introduction}
% Begin Section


\subsection{Purpose}
\label{sub:purpose}
% Begin SubSection
The purpose of this software requirement specification document is to provide a description of the
requirements needed to design the software for controlling the ParkFinder app. This app helps users
by providing a more efficient method for looking up parks and acquiring park information.
\newline\newline
The intended readers of this document include all of the project's stakeholders. This includes the
end-user, the software engineers, and the park authorities.
% End SubSection

\subsection{Scope}
\label{sub:scope}
% Begin SubSection
The software product being described in this document is called the ParkFinder app. This product
will have datasets of information about parks all over the world and will allow the client to use
search methods in order to find parks based on their desired attributes. The app is meant to be used
anywhere in the world, provided an Android or iOS device with the app installed. This provides
clients with an easier, faster, and more efficient way to look up parks and acquire information such
as the location, facilities, activities, and rentals that the parks provide.
% End SubSection

\subsection{Definitions, Acronyms, and Abbreviations}
\label{sub:definitions_acronyms_and_abbreviations}
% Begin SubSection
\begin{description}
    \item[\textbf{Experts}] A search criteria used for the identification of a particular park, or
    group of parks. ex. An expert could be for the rentals available at a park
    \item[\textbf{SRS}] Software requirements specification.
    \item[\textbf{API}] Application programming interface.
\end{description}

% End SubSection

\subsection{References}
\label{sub:references}
% Begin SubSection
N/A
% End SubSection

\subsection{Overview}
\label{sub:overview}
% Begin SubSection
The following section of this document, Overall Description, provides the reader with an overview of
many important aspects of ParkFinder. This includes information about the functionality of the app,
characteristics of the intended users, the constraints that will limit developer options, and any
assumptions or dependencies that can potentially change the requirements. The last two sections of
the SRS deal with both the Functional and Non-Functional Requirements. Included with the Functional
Requirements will be all the business events the system will need to handle and the viewpoints for
each event. The Non-Functional Requirements are divided into several categories such as requirements
for the look and feel of the system or the performance requirements.
% End SubSection

% End Section

\section{Overall Description}
\label{sec:overall_description}
% Begin Section

\subsection{Product Perspective}
\label{sub:product_perspective}
% Begin SubSection
Several web-services, such as Google or \url{www.ontarioparks. 	ca}, are available to assist with
locating parks. However these services are not available to an off-line user. The ParkFinder app
will allow its users to find parks that match their search criteria. More importantly, the app
will be able to perform the majority of its search functionalities without an Internet connection,
thus allowing the user the freedom to use the app whenever and wherever they want. As the ParkFinder
app only requires external resources for a subset of its functionality and no external applications
depend upon the ParkFinder app, it cannot be considered to be a component of a larger system.
% End SubSection

\subsection{Product Functions}
\label{sub:product_functions}
% Begin SubSection
\begin{enumerate}[a)]
	\item The product shall maintain a database of parks and associated attributes
    \item The app will provide a minimum of four methods for querying the park database, these
    methods	will be referred to as ``experts''
    \item One expert must be able to locate parks based on location, to do so
    an on-line mapping service will need to be utilized (when an Internet connection is available)
    \item Results from a users query shall be displayed via a centralized ``forum''
    \item \textcolor{red}{Information sent to and from the ``forum'' shall be encrypted to ensure
    privacy}
    \item The user can select a park from the query results and the ParkFinder app shall display
    more detailed information about that park, including a weather report (when an Internet
    connection is available)
\end{enumerate}
% End SubSection

\subsection{User Characteristics}
\label{sub:user_characteristics}
% Begin SubSection
The intended users of the ParkFinder app are people wishing to discover new parks that they have not
been to before as well as people completely new to visiting parks. It is expected that the primary
users will be adults however, visiting parks can be a family affair and thus children can also
be expected to use the ParkFinder. \textcolor{red}{Thus we expect the ParkFinder app to be used
by children no younger than that of a third grader (8-9 years old), adults, and the elderly. It is
expected that users will understand English, up to a grade three level.} Being an app solely
available on mobile devices, it can be expected that users will possess the bare minimum skill
required to operate a mobile device. Such skills would include being able to select buttons, use a
keyboard, and navigate menu screens.
% End SubSection

\subsection{Constraints}
\label{sub:constraints}
% Begin SubSection
The primary constraint limiting the development teams options is time. Due to chaotic scheduling
between this project and others, some minor functionality may not be implemented.
% End SubSection

\subsection{Assumptions and Dependencies}
\label{sub:assumptions_and_dependencies}
% Begin SubSection
For the use external services, such as weather updates or the use of a mapping service, it is
assumed that the user has an Internet connection. These services are also assumed to provide
accurate information to the user. Also assumed that the user will run the application on supported
android device. All information acquired about each park is assumed to be correct and up to date. 
% End SubSection

\subsection{Apportioning of Requirements}
\label{sub:apportioning_of_requirements}
% Begin SubSection
The addition of parks from all over the world may be delayed and just parks from Ontario will be
implemented in the initial version of the system.
% End SubSection

% End Section

\section{Functional Requirements}
\label{sec:functional_requirements}
% Begin Section
This section of the SRS contains all of the functional software requirements for the ParkFinder app.
This section enables designers to design a system satisfying those requirements, and the testers to
test that the system satisfies those requirements.\\
\textbf{Note: } The functional requirements are organized by business events (BE), then by
viewpoints (VP).

\begin{enumerate}[{BE}1.]
	\item The developer wants to change or remove an expert
	\begin{enumerate}[{VP\theenumi}.1]
		\item Developer
			\begin{enumerate}
				\item The system will allow the developer to swap or remove the expert
			\end{enumerate}
		\item Security
			\begin{enumerate}
				\item The system will check if the swap is being made by an authorized party
			\end{enumerate}
		\item User
            \begin{enumerate}
                \item N/A
            \end{enumerate}
        \item Manager
            \begin{enumerate}
                \item The manager will be asked to give permission for the swap
            \end{enumerate}
	\end{enumerate}

	\item \textcolor{red}{The user wants to search for parks that fit specified criteria}
    \label{BE_combined}
    \begin{enumerate}[{VP\theenumi}.1]
        \item Developer
            \begin{enumerate}
                \item N/A
            \end{enumerate}
        \item Security
            \begin{enumerate}
                \item The system will encrypt and then decrypt the user's input
                \item The system will encrypt and then decrypt the system's output
            \end{enumerate}
        \item User
            \begin{enumerate}
                \item \textcolor{red}{The system shall enable the user to choose different types of
                search criteria, such as: amenities, activities, rentals, park size, and seasonal
                dates}
                \item \textcolor{red}{Search results will be displayed, showing all parks that match
                the chosen criteria}
            \end{enumerate}
        \item Manager
            \begin{enumerate}
                \item N/A
            \end{enumerate}
    \end{enumerate}

    \item The user wants to view more information about a specific park
    \begin{enumerate}[{VP\theenumi}.1]
        \item Developer
            \begin{enumerate}
                \item N/A
            \end{enumerate}
        \item Security
            \begin{enumerate}
                \item The system will encrypt and then decrypt the user's input
                \item The system will encrypt and then decrypt the system's output
            \end{enumerate}
        \item User
            \begin{enumerate}
                \item The system shall give the user an overview of the park. Thus will include the
                highlights and popular attributes of the park, the address, phone number, size,
                website, and operational dates
                \item The system shall give the user information about all of the available
                amenities, activities, and rentals at the park
                \item The system will show the current weather conditions at the park
            \end{enumerate}
        \item Manager
            \begin{enumerate}
                \item N/A
            \end{enumerate}
    \end{enumerate}

    \item The user requests to view the location or locations of a selected park or several parks
    \begin{enumerate}[{VP\theenumi}.1]
        \item Developer
            \begin{enumerate}
                \item N/A
            \end{enumerate}
        \item Security
            \begin{enumerate}
                \item The system will encrypt and then decrypt the user's input
                \item The system will encrypt and then decrypt the system's output
            \end{enumerate}
        \item User
            \begin{enumerate}
                \item The system shall display the location of the park or parks on a map
            \end{enumerate}
        \item Manager
            \begin{enumerate}
                \item N/A
            \end{enumerate}
    \end{enumerate}

    \item The user wants to find the nearest 5 parks to their current location
    \begin{enumerate}[{VP\theenumi}.1]
        \item Developer
            \begin{enumerate}
                \item N/A
            \end{enumerate}
        \item Security
            \begin{enumerate}
                \item The system will encrypt and then decrypt the user's input
                \item The system will encrypt and then decrypt the system's output
            \end{enumerate}
        \item User
            \begin{enumerate}
                \item The system will display the closest 5 parks to the user's location
            \end{enumerate}
        \item Manager
            \begin{enumerate}
                \item N/A
            \end{enumerate}
    \end{enumerate}

\end{enumerate}
% End Section

\section{Non-Functional Requirements}
\label{sec:non-functional_requirements}
% Begin Section
\subsection{Look and Feel Requirements}
\label{sub:look_and_feel_requirements}
% Begin SubSection
\setcounter{myCounter}{0}

\subsubsection{Appearance Requirements}
\label{ssub:appearance_requirements}
% Begin SubSubSection
\begin{enumerate}[{LF}1. ]
    \setcounter{enumi}{\themyCounter}
    \item The application shall use the team logo displayed on team developed pages of the
    application. 
    \item The application shall use interactive displays that look like they can be interacted with.
    \setcounter{myCounter}{\theenumi}
\end{enumerate}
% End SubSubSection

\subsubsection{Style Requirements}
\label{ssub:style_requirements}
% Begin SubSubSection
\begin{enumerate}[{LF}1. ]
    \setcounter{enumi}{\themyCounter}
    \item After reading a brief description of the application, 90 percent of potential users shall
    feel that the application is trustworthy.
\end{enumerate}
% End SubSubSection

% End SubSection

\subsection{Usability and Humanity Requirements}
\label{sub:usability_and_humanity_requirements}
% Begin SubSection
\setcounter{myCounter}{0}

\subsubsection{Ease of Use Requirements}
\label{ssub:ease_of_use_requirements}
% Begin SubSubSection
\begin{enumerate}[{UH}1. ]
    \setcounter{enumi}{\themyCounter}
    \item 90 percent of a test panel with a third grade education shall successfully complete the
    use of three application experts within 20 minutes of using the application. 
    \item After a two-week period with the application, users shall exhibit an error rate of less
    than two percent.
    \item A feedback survey shall show that 85 percent of users with little English background shall
    successfully use one expert's functionality within 20 minutes. 
    \item 90 percent of a sample of people with a third grade education shall understand 99 percent
    of the language used in the application.      
	\setcounter{myCounter}{\theenumi}
\end{enumerate}
% End SubSubSection

\subsubsection{Personalization and Internationalization Requirements}
\label{ssub:personalization_and_internationalization_requirements}
% Begin SubSubSection
\begin{enumerate}[{UH}1. ]
    \setcounter{enumi}{\themyCounter}
    \item The application shall allow the user to choose their preferred background colour scheme
    for the application.
    \setcounter{myCounter}{\theenumi}
\end{enumerate}
% End SubSubSection

\subsubsection{Learning Requirements}
\label{ssub:learning_requirements}
% Begin SubSubSection
\begin{enumerate}[{UH}1. ]
    \setcounter{enumi}{\themyCounter}
    \item It shall take 20 minutes to learn how to use 90 percent of the application's functionality
    based on a test pool of people with a third grade education.
    \item All application functionality shall be easy to use without a tutorial or training for the
    application.
    \setcounter{myCounter}{\theenumi}
\end{enumerate}
% End SubSubSection

\subsubsection{Understandability and Politeness Requirements}
\label{ssub:understandability_and_politeness_requirements}
% Begin SubSubSection
\begin{enumerate}[{UH}1. ]
    \setcounter{enumi}{\themyCounter}
    \item Use of the application experts shall be intuitive as to what inputs these experts expect.
    \item The application shall hide its internal details from the user.
    \setcounter{myCounter}{\theenumi}
\end{enumerate}
% End SubSubSection

\subsubsection{Accessibility Requirements}
\label{ssub:accessibility_requirements}
% Begin SubSubSection
N/A
% End SubSection

\subsection{Performance Requirements}
\label{sub:performance_requirements}
% Begin SubSection
\setcounter{myCounter}{0}

\subsubsection{Speed and Latency Requirements}
\label{ssub:speed_and_latency_requirements}
% Begin SubSubSection
\begin{enumerate}[{PR}1. ]
    \setcounter{enumi}{\themyCounter}
	\item The application shall return information to the user within 2 seconds.
    \setcounter{myCounter}{\theenumi}
\end{enumerate}
% End SubSubSection

\subsubsection{Safety-Critical Requirements}
\label{ssub:safety_critical_requirements}
% Begin SubSubSection
\begin{enumerate}[{PR}1. ]
    \setcounter{enumi}{\themyCounter}
	\item The application shall not consume more than 500mW of power at any given time.
    \setcounter{myCounter}{\theenumi}
\end{enumerate}
% End SubSubSection

\subsubsection{Precision or Accuracy Requirements}
\label{ssub:precision_or_accuracy_requirements}
% Begin SubSubSection
\begin{enumerate}[{PR}1. ]
    \setcounter{enumi}{\themyCounter}
	\item The application shall display park ranges as a integer value. 
	\item The application shall display park size as an integer value. 
    \item The application shall display GPS coordinates up to two decimal places using decimal
    degree format. 
	\setcounter{myCounter}{\theenumi}
\end{enumerate}
% End SubSubSection

\subsubsection{Reliability and Availability Requirements}
\label{ssub:reliability_and_availability_requirements}
% Begin SubSubSection
\begin{enumerate}[{PR}1. ]
    \setcounter{enumi}{\themyCounter}
    \item The application's weather service and mapping service will only be available when the
    device has an Internet connection.
    \setcounter{myCounter}{\theenumi}
\end{enumerate}
% End SubSubSection

\subsubsection{Robustness or Fault-Tolerance Requirements}
\label{ssub:robustness_or_fault_tolerance_requirements}
% Begin SubSubSection
\begin{enumerate}[{PR}1. ]
    \setcounter{enumi}{\themyCounter}
	\item In the event of losing Internet connectivity, the user will still be able to use the application's experts. 
    \setcounter{myCounter}{\theenumi}
\end{enumerate}
% End SubSubSection

\subsubsection{Capacity Requirements}
\label{ssub:capacity_requirements}
% Begin SubSubSection
N/A
% End SubSubSection

\subsubsection{Scalability or Extensibility Requirements}
\label{ssub:scalability_or_extensibility_requirements}
% Begin SubSubSection
N/A
% End SubSubSection

\subsubsection{Longevity Requirements}
\label{ssub:longevity_requirements}
% Begin SubSubSection
\begin{enumerate}[{PR}1. ]
    \setcounter{enumi}{\themyCounter}
    \item The application is expected to operate and be available for download for at least 5 years
    without any budget constraint.
    \setcounter{myCounter}{\theenumi}
\end{enumerate}
% End SubSubSection

% End SubSection

\subsection{Operational and Environmental Requirements}
\label{sub:operational_and_environmental_requirements}
% Begin SubSection
\setcounter{myCounter}{0}

\subsubsection{Expected Physical Environment}
\label{ssub:expected_physical_environment}
% Begin SubSubSection
\begin{enumerate}[{OE}1. ]
    \setcounter{enumi}{\themyCounter}
	\item The application can be used anywhere on any supported Android device.
    \item The application location-based services will only be available where an Internet
    connection is present.
    \setcounter{myCounter}{\theenumi}
\end{enumerate}
% End SubSubSection

\subsubsection{Requirements for Interfacing with Adjacent Systems}
\label{ssub:requirements_for_interfacing_with_adjacent_systems}
% Begin SubSubSection
\begin{enumerate}[{OE}1. ]
    \setcounter{enumi}{\themyCounter}
    \item The application will interface with a mapping service, weather service and the device
    global positioning system.
    \setcounter{myCounter}{\theenumi}
\end{enumerate}
% End SubSubSection

\subsubsection{Productization Requirements}
\label{ssub:productization_requirements}
% Begin SubSubSection
\begin{enumerate}[{OE}1. ]
    \setcounter{enumi}{\themyCounter}
	\item The application will be easy to install through the application store service.
    \setcounter{myCounter}{\theenumi}
\end{enumerate}
% End SubSubSection

\subsubsection{Release Requirements}
\label{ssub:release_requirements}
% Begin SubSubSection
\begin{enumerate}[{OE}1. ]
    \setcounter{enumi}{\themyCounter}
	\item The application will be available to install through the application store service.
    \setcounter{myCounter}{\theenumi}
\end{enumerate}
% End SubSubSection

% End SubSection

\subsection{Maintainability and Support Requirements}
\label{sub:maintainability_and_support_requirements}
% Begin SubSection
\setcounter{myCounter}{0}

\subsubsection{Maintenance Requirements}
\label{ssub:maintenance_requirements}
% Begin SubSubSection
\begin{enumerate}[{MS}1. ]
    \setcounter{enumi}{\themyCounter}
    \item Application updates will be pushed through the application store services, annually, to
    all devices with installed instances of the app.
    \item Any application security vulnerabilities will be patched immediately and pushed to the
    application store service.
    \setcounter{myCounter}{\theenumi}
\end{enumerate}
% End SubSubSection

\subsubsection{Supportability Requirements}
\label{ssub:supportability_requirements}
% Begin SubSubSection
\begin{enumerate}[{MS}1. ]
    \setcounter{enumi}{\themyCounter}
	\item The application shall only run on supported Android devices.
    \setcounter{myCounter}{\theenumi}
\end{enumerate}
% End SubSubSection

\subsubsection{Adaptability Requirements}
\label{ssub:adaptability_requirements}
% Begin SubSubSection
\begin{enumerate}[{MS}1. ]
    \setcounter{enumi}{\themyCounter}
    \item The application shall be modular enough to make it easier for future feature additions and
    platform migrations.
    \setcounter{myCounter}{\theenumi}
\end{enumerate}
% End SubSubSection

% End SubSection

\subsection{Security Requirements}
\label{sub:security_requirements}
% Begin SubSection
\setcounter{myCounter}{0}

\subsubsection{Access Requirements}
\label{ssub:access_requirements}
% Begin SubSubSection
\begin{enumerate}[{SR}1. ]
    \setcounter{enumi}{\themyCounter}
    \item \textcolor{red}{Expert information and modification will only be accessible by system
    administrators, such as the managers and developers.}
    \setcounter{myCounter}{\theenumi}
\end{enumerate}
% End SubSubSection

\subsubsection{Integrity Requirements}
\label{ssub:integrity_requirements}
% Begin SubSubSection
\begin{enumerate}[{SR}1. ]
    \setcounter{enumi}{\themyCounter}
    \item \textcolor{red}{The application shall prevent incorrect data from being introduced.}
    \setcounter{myCounter}{\theenumi}
\end{enumerate}
% End SubSubSection

\subsubsection{Privacy Requirements}
\label{ssub:privacy_requirements}
% Begin SubSubSection
\begin{enumerate}[{SR}1. ]
    \setcounter{enumi}{\themyCounter}
    \item The application shall encrypt all user data. Any user data including searches will be kept
    confidential.
    \setcounter{myCounter}{\theenumi}
\end{enumerate}
% End SubSubSection

\subsubsection{Audit Requirements}
\label{ssub:audit_requirements}
% Begin SubSubSection
N/A
% End SubSubSection

\subsubsection{Immunity Requirements}
\label{ssub:immunity_requirements}
% Begin SubSubSection
\begin{enumerate}[{SR}1. ]
    \setcounter{enumi}{\themyCounter}
    \item The application shall be immune to any manual attacks such as man-in-the-middle as well as
    automated attacks from Trojan horses and malicious scripts.
    \setcounter{myCounter}{\theenumi}
\end{enumerate}
% End SubSubSection

% End SubSection

\subsection{Cultural and Political Requirements}
\label{sub:cultural_and_political_requirements}
% Begin SubSection
\setcounter{myCounter}{0}

\subsubsection{Cultural Requirements}
\label{ssub:cultural_requirements}
% Begin SubSubSection
\begin{enumerate}[{CP}1. ]
    \setcounter{enumi}{\themyCounter}
    \item The application will remain culturally and religiously sensitive.
    \setcounter{myCounter}{\theenumi}
\end{enumerate}
% End SubSubSection

\subsubsection{Political Requirements}
\label{ssub:political_requirements}
% Begin SubSubSection
\textcolor{red}{N/A}
% End SubSubSection

% End SubSection

\subsection{Legal Requirements}
\label{sub:legal_requirements}
% Begin SubSection
\setcounter{myCounter}{0}

\subsubsection{Compliance Requirements}
\label{ssub:compliance_requirements}
% Begin SubSubSection
\begin{enumerate}[{LR}1. ]
    \setcounter{enumi}{\themyCounter}
	\item The application shall comply with the regional safety and confidentiality requirements.
    \setcounter{myCounter}{\theenumi}
\end{enumerate}
% End SubSubSection

\subsubsection{Standards Requirements}
\label{ssub:standards_requirements}
% Begin SubSubSection
\begin{enumerate}[{LR}1. ]
    \setcounter{enumi}{\themyCounter}
    \item The application shall comply with Android standards including resolution, screen size, and
speed limitations.
    \setcounter{myCounter}{\theenumi}
\end{enumerate}
% End SubSubSection

% End SubSection

% End Section

\newpage
\appendix
\section{Division of Labour}
\label{sec:division_of_labour}
% Begin Section

\begin{table}[htbp]
\vspace{-0.06in}
\begin{center}
\setlength{\extrarowheight}{4.0pt}
\begin{tabular}{m{0.3\textwidth} m{0.2\textwidth} m{0.3\textwidth}} 
\hline
\textbf{Contributions} & \textbf{Name} & \textbf{Signature}\\
\hline
Sections
\ref{sub:operational_and_environmental_requirements}-\ref{sub:cultural_and_political_requirements} &
Abdul Ahad & \\
\hline
Sections
\ref{sub:purpose}-\ref{sub:scope}~\&~\ref{sec:functional_requirements} & Salma Belal & \\
\hline
Sections
\ref{sec:introduction}~\&
\ref{sub:assumptions_and_dependencies}-\ref{sub:apportioning_of_requirements} & Josh Chatten & \\
\hline
Sections
\ref{sub:look_and_feel_requirements}-\ref{sub:performance_requirements} & Nathanael Jordan  & \\
\hline
Sections
\ref{sub:product_perspective}-\ref{sub:constraints} & Robert Stuart & \\
\hline
\end{tabular}
\end{center}
\label{divOfLabour}
\end{table}

% End Section

\color{red}
\section{Change Log}
\label{sec:change_log}
% Begin Section

\subsection{March 4, 2016 - Re-submission with Deliverable 2}
\label{sub:d2}
% Begin Subsection

The following changes were made due to feedback from the teaching assistants:
\begin{itemize}
    \item \textbf{\nameref{sub:product_functions}} Experts being separate and easily swappable
    removed
    \item \textbf{\nameref{sub:product_functions}} Modified encryption/decryption statement to
    remove the idea of the user communicating with the experts
    \item \textbf{\nameref{sub:user_characteristics}} Explicitly stated the expected ages of the
    users as well as the expected target audience language (English)
    \item \textbf{\nameref{sec:functional_requirements}} Combined previously separated business
    events for each Expert type into a single business event (see BE\ref{BE_combined})
    \item \textbf{\nameref{sec:functional_requirements}} Removed the Geographical viewpoint from all
    business events
    \item \textbf{\nameref{ssub:access_requirements}} Corrected interpretation of the requirement
    \item \textbf{\nameref{ssub:integrity_requirements}} Corrected interpretation of the requirement
    \item \textbf{\nameref{ssub:political_requirements}} Removed unneeded requirement
\end{itemize}

% End Subsection

% End Section
\color{black}

\end{document}
%------------------------------------------------------------------------------